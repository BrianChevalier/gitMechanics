% Basic document class
\documentclass[landscape, twocolumn, 12pt]{article}

% General public packages loaded here:
\usepackage{amsmath}
\usepackage{amssymb}
\usepackage{graphicx}
\usepackage{caption}
\usepackage{float}
\usepackage{pdfpages}
\usepackage{gensymb}
\usepackage{tikz}
\usetikzlibrary{math}

% Custom .sty files loaded here:
\usepackage{../../TitlePage}
\usepackage{../../mathshortcuts}

% Title Page commands
% Check TitlePage.sty for documentation

\title{Taylor Series}
\subtitle{Approximating Functions with Polynomials}
\coursenum{CEE384}
\coursetitle{Numerical Methods}
\university{Arizona State University}

%\titleimage{}
\author{Brian Badahdah}
\date{\today}

%-------------------------------------------
\begin{document}

\maketitle
\section{Motivation}

There are many functions that are just a bit annoying. Instead of dealing with the actual function we want to deal with a much simpler polynomial approximation of that function. For instance, the familiar trigonometric functions have known values at discrete points but calculating the cosine of a value not at a particular point is not directly possible without a numerical approximation.

We will begin by looking at the sin function. Essentially we want to make the function sin equal to some polynomial

\begin{equation}
  \sin(x)=a_0 + a_1x+a_2x^2 + a_3x^3
\end{equation}

We have two problems. 1) we don't currently have a value for $x$ and 2) We don't have enough equations to determine the coefficients in the Taylor polynomial.

Problem one is pretty easy to solve. We just need to pick a point, $x$, to center our approximation about. This should be a convenient point that we know the value of the function at. $x=0$ will do for this example.

The second problem can be solved by generating more equations.

%---------------------------------------------
%\clearpage
%\glsaddall
%\printglossary[title={Variables Definitions}]

\newpage
\bibliography{../../assets/bibtex}
\addcontentsline{toc}{section}{References}
\bibliographystyle{apalike}
\end{document}
