\section{Making your Contribution}

% TO DO

Before you begin, you will need: a way to use git (command line or GUI option), a way to compile \LaTeX{} documents (command line or GUI option) and a github account.

The following steps are the steps you need to take to make your first contribution to the gitMechanics repository:
\begin{greenfig}
\begin{todolist}
	\item Fork the main gitMechanics repo
	\item Clone the fork from your account to your local machine
	\item Make changes to the source code and recompile your PDFs as needed
	\item Commit changes to your local repo and push changes to your github server repo
	\item Once you want to make a change to the website, go to github and make a pull request on the main repo
\end{todolist}
\end{greenfig}

\subsection{How the repo is organized}

Figure \ref{Fig:dirtree} shows the typical directory of a courses files and will be used to explain how the repo is organized.

% SOURCE:
% https://tex.stackexchange.com/questions/5073/making-a-simple-directory-tree
\begin{figure}
\begin{center}
\begin{forest}
for tree={
    font=\ttfamily,
    text=black,
    text width=3.5cm,
    minimum height=0.75cm,
    if level=0
      {fill=mattegreen}
      {fill=mattegreen},
    rounded corners=4pt,
    grow'=0,
    child anchor=west,
    parent anchor=south,
    anchor=west,
    calign=first,
    edge={mattegreen,rounded corners,line width=1pt},
    edge path={
      \noexpand\path [draw, \forestoption{edge}]
      (!u.south west) +(7.5pt,0) |- (.child anchor)\forestoption{edge label};
    },
    before typesetting nodes={
      if n=1
        {insert before={[,phantom]}}
        {}
    },
    fit=band,
    s sep=15pt,
    before computing xy={l=15pt},
  }
[gitMechanics
	[CEE321
		[index.md]
		[Direct-stiffness
			[Main.tex]
    	[Main.pdf]
    	[Figures
    		[Figure-1.pdf]
    		[Figure-2.png]
    	]
    	[Sections
    		[Section-1.tex]
    		[Section-2.tex]
    	]
		]
	]
]
\end{forest}
\end{center}
\caption{Directory of a typical course}
\label{Fig:dirtree}
\end{figure}


