\section{Introduciton to \LaTeX{}}
\LaTeX{} (pronounced lay-teck) is a free, open-source high-quality document preparation system (this document is written and compiled with \LaTeX{}!). \TeX{} was written and designed by Donald Knuth in 1978. \LaTeX{} builds on top of the work of Knuth providing more functionality through a more advanced set of macros.

\subsection{Vernacular}

\begin{description}
	\item[Macro] A macro is a \LaTeX{} command that can take input and does something with it. You can define any number of custom macros (also called commands) that can speed up your document preparation and make your style more consistent.
	\item[Package] A package is a set of predefined \LaTeX{} macros packaged up for your use. Users can create their own packages. gitMechanics relies on a few custom packages.
	\item[Document class] There are many document classes that you can use, and you can even define your own classes. Classes include: article, standalone, beamer (slide presentation format), report, book, etc. Each class provides a set of commands to typeset your work.
	\item[Environments] Text in an environment is processed in a particular way by the compiler. Inside an environment particular types of commands are available for use. The entire body of the document is wrapped in the document environment. There are also list environments, math environments, floating environments, and of course you can define custom environments.
\end{description}

\subsection{Getting \LaTeX{} on Your Machine}
Your first step is to install a distribution of \LaTeX{}. This will be different depending on your platform, however it is available on all platforms (you can use the Linux distribution on ChromeOS. On iOS you can get a dedicated app, I reccomend TeX Writer). The following URL includes links on where to install the different distributions.

\begin{center}
	\url{https://www.latex-project.org/get/}
\end{center}

Most also come with a GUI editor and built in methods for compiling your documents so you don't have to touch the command line. Each editor will be slightly different so you'll have to spend a while getting to know your editor. Note that you can also opt to use your own text editor and compile from the command line using pdflatex. Regardless, make sure your text editor has good code completion support for \LaTeX{} otherwise it will make learning commands a nightmare.


\subsection{Compiling Your First Document}

The most basic document needs two things. First you have to define your ``document class''. As discussed before, there are many classes. Document classes can also take optional inputs in square brackets before the actual document class. This document is an article class document and the options are landscape, twocolumn, and 12pt. The options must be separated by commas. The example code below typsets a blank document. Note that the article declaration is on the second line, which is completely valid, but is also done to fit within the column margins. Also note that your document will be within the document environment, and that the \% sign indicates a comment.

\begin{center}
\begin{latexcode}
\documentclass[landscape, twocolumn, 12pt]
{article}

\begin{document}
% Your document goes here
\end{document}
\end{latexcode}
\end{center}

\subsection{Document Structure}



\subsection{Loading Packages}
Standard \LaTeX{} is good, however, you will likely need to load additional packages to extend the language. You can import a package like the following:

\begin{center}
\begin{latexcode}
\usepackage{amsmath}
\end{latexcode}
\end{center}
You will want to typically use the AMSMath package since it adds many commonly used math symbols. You can find full documentation on the package at:

\begin{center}
	\url{https://www.ctan.org/pkg/amsmath}
\end{center}

The Comprehensive \TeX{} Archive Network (CTAN) has many packages and documentation for packages that are typically already included with your \LaTeX{} distribution. Most packages have the full documentation in PDF format available on CTAN.


\subsection{Equations in \LaTeX{}}
If there's one thing \LaTeX{} is known for it's typesetting math and equations. We're going to get into that now. For math to be added to your document you have to be in some sort of math mode. This means you could be in a math environment or have your math inline with delimiters. 

Math environments include: align, equation, or a few more. Equations in this environment will automatically be centered and numbered. You can suppress numbering of these environments by adding ``*'' after the environment name. The following equation is typeset using the equation environment:

\begin{equation}
	\int_{0}^{L} f(x) dx
\end{equation}

\begin{center}
\begin{latexcode}
\begin{equation}
\int_{0}^{L} f(x) dx
\end{equation}
\end{latexcode}
\end{center}

Note that the braces around the limits of integration are not strictly needed for single character limits, however it is best practice to keep them in braces to avoid confusion and be consistent for more complicated equations.

The next important math environment is the align environment. The align environment allows typesetting multiple consecutive equations aligned. The following is an example of that:

\begin{align}
y &= mx + b\\ 
  &= \frac{\Delta y}{\Delta x}x + b
\end{align}

The syntax looks like:

\begin{center}
\begin{latexcode}
\begin{align}
y &= mx + b\\ 
  &= \frac{\Delta y}{\Delta x}x + b
\end{align}
\end{latexcode}
\end{center}

Things to note: the equations are aligned based on the location of the ``\&'' and new lines are entered with ``\textbackslash''. Greek letters and other commands also require a space after the command name otherwise it will throw an error.

\subsubsection{Lableing and Referencing}

Hyperlinking

\subsubsection{Greek Letters}
Most Greek letters can be included in math mode typically by typing ``\textbackslash'' followed by the name of the letter such as typesetting the letter beta: $\beta$. The following PDF lists many Greek letters and math mode symbols.

\begin{center}
	\url{https://wch.github.io/latexsheet/latexsheet.pdf}
\end{center}



\subsubsection{Operators}















