\section{Introduciton to Jekyll}

Jekyll is a blog-aware static site generator. Github supports hosting websites with custom domains using the Jekyll blogging platform. The major benefit to this (other than that it is free), is that anytime the master branch of the github repo is updated via a commit, github automatically rebuilds the site using Jekyll and makes it live. This is great because all of the source documents can be stored in the same repository as the source files that build the website. It also has the added benefit of automatically rebuilding itself if someone makes a pull request on the repo and it is accepted.

%% Setting up a local jekyll server

%% My remote building set-up

% Set-up
% ufw: Uncomplicated firewall
% Open up port 4000
% https://wiki.archlinux.org/index.php/Uncomplicated_Firewall

% Install ruby and Jekyll

% Add ruby to your path
% PATH=$PATH:/root/.gem/ruby/2.5.0/bin/

% To build the site
% bundle exec jekyll serve --host=[Your Server is Address]

% Connect to the server by typing the server IP into a browser followed by :4000