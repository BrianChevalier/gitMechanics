\section{Drawing in \LaTeX{}}

\LaTeX{} produces truly stunning PDF output, but as the great philosopher, Beyonc\'e once said, ``Pretty Hurts''.

\paragraph{GUI Options}
There are some options to avoid manually marking up drawings. One method is to simply produce figures and drawings in another application, and export as a PDF, or if you like the pixelated compression artifacts, JPEG. You can then save the file in the Figures subdirectory of your project and include something like the following:

\begin{center}
\begin{latexcode}
\begin{figure}[H]
\centerline{
\includegraphics[width=\columnwidth]
{Figures/FIGNAME}}
\caption{CAPTION}
\label{fig:LABEL}
\end{figure}
\end{latexcode}
\end{center}

You can also use a GUI editor that will output to markup formats and include those directly into your source code. You can export to TikZ format which we will discuss in the next section. A great option for this is GeoGebra:

\begin{center}
	\url{https://www.geogebra.org}
\end{center}

\subsection{TikZ}
While including external images works, it can get out of control pretty quickly if you have many figures. It also eliminates the possibility of using a set of macros to keep drawings consistent throughout many documents. Using generated TikZ still runs into this problem. Therefore, the only good option is to learn TikZ markup. For gitMechanics, I have already begun building a library of macros that will hopefully make the processs more straight-forward and consistent between documents.

The following site has many example TikZ examples:

\begin{center}
\url{http://www.texample.net/tikz/examples/}
\end{center}



\subsubsection{TikZ Math}
The TikZ Math library is a great way to use math to better geometrically describe your diagrams. The direct link to the documentation can be found at:

\begin{center}
\url{http://ctan.mirrors.hoobly.com/graphics/pgf/base/doc/pgfmanual.pdf\#page640}
\end{center}

You must load the tikzmath library using the following:

\begin{center}
\begin{latexcode}
\usetikzlibrary{math}
\end{latexcode}
\end{center}

Then you can define variables and evaluate math expressions in the tikzmath command. Each statement \textbf{must} be terminated by a semi-colon, and the variable should not be an existing variable used in your \LaTeX{} file or used packages.

