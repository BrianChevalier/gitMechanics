\section{The First Law}
The total energy flowing in and out of the system is equal to the total change in internal energy change in the system. That is mathematically written as:

\begin{align}
	Q-W&=\Delta E\\ 
	(Q_{in}-Q_{out}) - (W_{out}-W_{in}) &=\Delta E
\end{align}


Some definitions:

\begin{description}
	\item[Isothermal] Constant temperature
	\item[Isobaric] Constant pressure
	\item[Isochoric] Constant volume
\end{description}


\section{Conservation of Mass}

A control volume is a volume used to model and track physical processes. It is similar to a free body diagram. The control surface is the surface surrounding the volume.



The general conservation of mass equation is:

\begin{align}
\frac{d}{dt}\int_{cv} \rho dV
+
\int_{cs}\rho(\vec{V}\cdot\vec{n})dA
=
0
\end{align}

Mass balance for a steady flow process:

\begin{align}
\frac{dm_{cv}}{dt}
=
\sum_{in} \dot{m}
-
\sum_{out} \dot{m}
\end{align}

\begin{align}
\dot{m}_1      &= \dot{m}_2\\
\rho_1 V_1 A_1 &= \rho_2 V_2A_2
\end{align}

For incompressible fluids the density is constant, therefore the density eliminates from both sides.

Total energy for a flowing fluid, the energy per unit mass is:

\begin{align}
\theta =
\underbrace{PV}_{\text{flow energy}}
+
\underbrace{u}_{\text{internal energy}}
+
\underbrace{\frac{V^2}{2}}_{\text{kinetic energy}}
+
\underbrace{gz}_{\text{potential energy}}
\end{align}

Amount of energy transport:

\begin{align}
E_{mass} &= m\theta\\
         &= m(h+\frac12V^2+gz)
\end{align}

The rate of energy transport:


\begin{align}
\dot{E}_{mass} &= \dot{m}\theta\\
               &= \dot{m}(h+\frac12V^2+gz)
\end{align}


Applying energy balance for multiple a control volume with multiple inlets or outlets:

\begin{align}
\dot{E}_{in} &= \dot{E}_{out}\\ 
\dot{Q}_{in} + \dot{W}_{in} + \sum_{in}\dot{m}\theta &=
\dot{Q}_{out} + \dot{W}_{out} + \sum_{out}\dot{m}\theta
\end{align}

\subsubsection{Diffuser}
A diffuser is a device that reduces the velocity of a flowing fluid. The governing equation for a diffuser is:

\begin{align}
	h_2 = h_1 + \frac{V_1^2}{2}
\end{align}

\subsubsection{Nozzle}
A nozzle is a device that increases the velocity of a flowing fluid. The governing equation is:

\begin{align}
	0&=\dot{Q} + \dot{m}
	\left[
	(h_1-h_2)
	+
	\frac{V_1^2-V_2^2}{2}
	\right]
\end{align}

\subsubsection{Turbine/Compressor}

\begin{align}
	\dot{W} = \dot{m}(h_1-h_2)
\end{align}


\subsubsection{Throttling Valve}
Reduces pressure
\begin{align}
	\dot{m}(h_1-h_2)&=0\\ 
	h_1=h_2
\end{align}


\section{Boundary Work}
Boundary work is the work done on a system that increases the size of a control volume.

Boundary work at constant pressure:
\begin{align}
	W_b &= \int_1^2 PdV\\ 
	&= P_0 \int_1^2\\ 
	&=P_0(V_2-V_1)\\
	&=mP_0(v_2-v_1)
\end{align}



Boundary work at constant pressure:
\begin{align}
	W_b &= \int_1^2 PdV\\ 
	&= 0
\end{align}

Boundary work at constant temperature:

\begin{align}
	W_b &= \int_1^2 PdV\\ 
	&= P_1V_1\ln(\frac{V_2}{V_1})
\end{align}

