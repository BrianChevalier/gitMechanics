
\section{Beam Analysis}
\subsection{Ultimate Capacity}
Concrete beams are analyzed at the \textit{ultimate state} to determine the ultimate strength or ultimate capacity of the beam. When a concrete beam is under a distributed load top fiber in compression bottom fiber in compression. The bottom part of the beam below the neutral axis must be reinforced with steel to prevent cracking from the bottom fibers propagating upward causing failure of the whole beam. 

At this state the concrete at the top of the beam is \textit{just} beginning to crush. ACI states that the compressive strain of concrete occurs at $\ep_c=\ecu=0.003$. This failure condition will be used to determine the needed tensile reinforcement. The strain profile of the cross section is linear. The compressive side of the beam has a non-linear stress profile. The tensile side of the beam is assumed to only be supported by the tensile force of the reinforcing steel.


\subsection{The Whitney Stress Block}
The \textit{Whitney Stress Block} also referred to as the \textit{equivalent stress block} is the solution for the non-linear stress profile of concrete in compression.

The parameter \betao is calculated by equation \ref{Eq:betao} which is taken from ACI Table 22.2.2.4.3, where \fc  is taken to be in units of psi.

\begin{align}
	\betao
	=
	\begin{cases}
		0.85 & \fc \leq 4000 \\
		0.85-0.05\left[\frac{\fc-4000}{1000}\right] & 4000 \leq\fc\leq 8000 \\
		0.65 & \fc > 8000
	\end{cases}
	\label{Eq:betao}
\end{align}



\subsection{The Balance Condition}

The balance condition is the condition which the concrete crushes at the same time that the steel yields. To solve for the neutral axis at the balance condition we use similar triangles.

\begin{align}
	\frac{c}{\ecu} &= \frac{d-c}{\es}\\ 
	c \es &= \ecu(d-c)\\ 
	c \es &= d\ecu -c\ecu
\end{align}

\gbox{
	c_{bal} &= \frac{d\ecu}{\es+\ecu}
}

Using $c_{bal}$ we can calculate the area of the steel \As at the balance condition. This is the area of steel needed for the steel to yield at the same time that the concrete crushes.

\gbox{
	A_{s,bal} &= \frac{0.85\fc\betao c_{bal}\bb}{\fy}
	\label{Eq:Asbal}
}

Next, we find the reinforcement ratio of the balance condition. The reinforcement ratio is the ratio of the area of steel to the area above the tension steel ($\bb\dd$). Therefore to calculate the reinforcement ratio take equation \ref{Eq:Asbal} as the area of steel. This results in equation \ref{Eq:rhobal}.

\gbox{
	\rho_{bal} &= 
	\frac{0.85\fc\betao}{\fy}
	\left[
		\frac{\ecu}{\es+\ecu}
	\right]
	\label{Eq:rhobal}
}

This equation can be speacialized by plugging in the value of \ecu and multiply the numerator and denominator by the modulus of elasticity of steel $E_s=29000$ ksi the result is shown in equation \ref{Eq:rhobal_ksi}. The value for \fy that is plugged in must be in ksi. 

\begin{align}
	\rho_{bal}
	&= 
	\frac{0.85\fc\betao}{\fy}
	\left[
		\frac{87}{87+\fy}
	\right]
	\label{Eq:rhobal_ksi}
\end{align}






